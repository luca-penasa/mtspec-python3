\documentclass{article}

\usepackage[intlimits]{amsmath}
\usepackage{amsfonts}
\usepackage{amscd}
\usepackage{amssymb}
\usepackage{natbib}


\ifx\pdfoutput\undefined
\usepackage[dvips]{graphicx}
\else
\usepackage[pdftex]{graphicx}
\pdfcompresslevel=9
\usepackage{epstopdf}
\fi

\usepackage{rotating}
\usepackage{lscape}

\title{Documentation for \\
	A Fortran 90 library for multitaper spectrum analysis}
\date{}
\begin{document}
\DeclareGraphicsExtensions{.pdf,.gif,.jpg, .eps}

\maketitle

\section{Introduction}

The present documentation is for the Fortran 90 Library for multitaper spectral analysis as presented in 
\begin{quotation}
Prieto, G. A., R. L. Parker and F. L. Vernon (2008). A Fortran 90 library for multitaper spectrum analysis. Submitted to Computers and Geosciences, February 2008.
\end{quotation}
It is intended to be a comprehensive and easy to use collection of subroutines for univariate and multivariate spectral analysis. There are two main methods implemented here, the original adaptive multitaper spectrum \citep{djt_82} and the sine multitaper method \citep{riedel_sidorenko_95}. 

\section{Requirements}
The first requirement is to have a Fortran 90 or 95 compiler available in your machine. If you want to be able to obtain figures from the programs, {\it gplot} needs to be installed as well. This library has been tested using the Absoft Compiler in a Mac OS X system, and using the Sun Compiler in the Unix environment. More recently I am using the Intel compiler on an Intel Mac machine. Since we are taking advantage of features in Fortran 90 such as optional arguments and modules not available in F77, a F90/F95 compiler is needed. Most of the things you can do in F77 can be compiled by F90, so there is no reason not to use F90 instead (of course this is a personal opinion, not necessarily shared by my co-authors.). 

A second requirement is that you have to edit the Makefile a little bit to adjust to your specific compiler options. There are small differences between compilers, especially the option to look for modules. You edit the Makefile to be able to search the modules used by this library. This release has a set of short programs to reproduce the figures presented in the paper \citep{prieto_08a}.

\section{Univariate subroutines}
In this documentation the main subroutines calls are explained. 

\subsection{\texttt{mtspec}}
This subroutine uses Thomson's spectrum estimation method \citep{djt_82} and outputs the power spectral density (PSD), and if requested performs F-test for line components, returns such as a jackknife $95\%$ confidence interval, the $y_k$'s, which contain the phase information, the weights of the different eigenspectra, etc. 

The basic way to call this subroutine is
\begin{quote}
\begin{verbatim}
call mtspec(npts,dt,x,tbp,kspec,nf,freq,spec)
\end{verbatim}
\end{quote}
A description of all the variables is presented in Table \ref{tab:mtspec_arg}. This particular subroutine performs all calculations in double precision, but accepts both single as well as double precision arguments. Note that if double precision is requested, all variables need to be double precision (except integers).

Taking advantage of Fortran 90 optional arguments, the user can request additional outputs or different form of calculations (e.g., F-test line reshaping, different weighting, quadratic multitaper, etc.).

If for example the user would also like to have the jackknife error estimate, the  call would be:
\begin{quote}
\begin{verbatim}
call mtspec(npts,dt,x,tbp,kspec,nf,freq,spec,err=jack)
\end{verbatim}
\end{quote}
which will return the appropriate jackknife estimate in the array \texttt{jack}, featuring the $5\%$ and the $95\%$ confidence intervals. A more complete description of the optional parameters is presented in Table \ref{tab:mtspec_opt}. 

\begin{table}[htdp]
\caption{\texttt{mtspec} mandatory arguments}
\begin{center}
\begin{tabular}{|c|c|c|c|l|}
\hline
\textbf{Var} &	\textbf{Type}	&  \textbf{I/O}	&	\textbf{Dimension}	&	\textbf{Description}		\\
\hline	\hline
npts		& 	integer	& input &	1x1				&	number of points in time series 		\\
dt		&	real		& input &	1x1				&	the sampling interval				\\
x		&	real		& input &	npts x 1			&	the real time series				\\
tbp		&	real		& input &	1x1				&	the time-bandwidth product		\\
kspec	&	integer	& input &	1x1				&	the number of tapers to use		\\
nf		&	integer	& input &	1x1				&	number freq. bins 		\\
freq		&	real		& output &	nf x 1			&	the real frequency vector			\\
spec		&	real		& output &	nf x 1			&	the spectrum estimate			\\
\hline
\end{tabular}
\end{center}
\label{tab:mtspec_arg}
\end{table}%

\begin{table}[!h]
\caption{\texttt{mtspec} optional arguments}
\begin{center}
\begin{tabular}{|c|c|c|c|l|}
\hline
\textbf{Var} &	\textbf{Double}	& \textbf{I/O}&	\textbf{Dimension}	&	\textbf{Description}		\\
\hline \hline
yk		&	complex	& output &	npts x kspec		&	the eigencoefficients 	\\
wt		&	real	& output &	nf x kspec		&	weights 			\\
err		&	real	& output &	nf x 2 		&	$95\% $ jackknife c.i.	\\
se		&	real	& output &	nf x 1			&	ndf for each freq bin				\\
sk		&	real	& output &	nf x kspec			&	the eigenspectra (${y_k}^2$)		\\
verb		&	integer	& input &	1 x 1			&	verbal option \\
		&		& 	 &			&	0 - default, no verbose	\\
		&		& 	 &			&	1 - print various steps	\\
qispec		&	integer	& input &	npts x kspec		&	Use quadratic method			\\
		&		&  &			&	0 - default, standard method	\\
				&		&  &			&	1 - quadratic method	\\
adapt	&	integer	& input &	1 x 1			&	adaptive multitaper		\\
		&		&  &			&	0 - default, adaptive MT	\\
				&		&  &			&	1 - constant weights	\\
rshape	&	integer	& input &	1 x 1			&	F-test periodic components		\\
		&		&  &					&	1 - remove lines	\\
				&		&  &			&	2 - remove lines around 60 Hz. 	\\
								&		&  &			&	other - reshape spectrum \\
fcrit		&	real	& input &	1 x 1				&	F-test threshold probability 			\\
								&		&  &			&	rshape present \\
fstat	&	real	& output &	nf				&	F statistics  (rshape present)	\\
\hline
\end{tabular}
\end{center}
\label{tab:mtspec_opt}
\end{table}%

\subsection{\texttt{mtspec} with zero padding}
In many cases it is advantageous to pad the data series with zeros. It that case, it is not recomended to pad the data before tapering, and I prefer to taper and then pad with zeros. This is performed internally inside the subroutine by calling
\begin{quote}
\begin{verbatim}
call mtspec(npts,nfft,dt,x,tbp,kspec,nf,freq,spec)
\end{verbatim}
\end{quote}
where $\mathtt{nfft} \ge \mathtt{npts}$. All other optional arguments can still be added as explained above.

Be aware that the number of frequency points \texttt{nf}, has to be either $\mathtt{npts}/2 + 1$ or $\mathtt{nfft}/2+1$, depending on the type of call. The output will then be a frequency vector \texttt{freq(nf)} containing all the frequency bins, and the spectrum \texttt{spec(nf)} with the power spectrum of the time series. 

\subsection{\texttt{sine\_psd}}
The subroutine is in charge of estimating the adaptive sine multitaper as in \citet{riedel_sidorenko_95}. If requested it will also calculate the sine multitaper with a constant number of tapers. The basic call is:
\begin{quote}
\begin{verbatim}
call sine_psd (npts,dt,x,ntap,ntimes,fact,nf,freq,spec)
\end{verbatim}
\end{quote}
The arguments are explained in Table \ref{tab:sine_arg}. If the user wants a constant number of tapers, just have the integer \texttt{ntap} be greater than 1. This parameter overrides the following two, so if an adaptive sine multitaper is required, \texttt{ntap} = 0.

\begin{table}[htdp]
\caption{\texttt{sine\_psd} mandatory arguments}
\begin{center}
\begin{tabular}{|c|c|c|c|l|}
\hline
\textbf{Var} &	\textbf{Type}	&  \textbf{I/O}	&	\textbf{Dimension}	&	\textbf{Description}		\\
\hline	\hline
npts		& 	integer	& input &	1x1				&	number of points in time series 		\\
dt		&	real		& input &	1x1				&	the sampling interval				\\
x		&	real		& input &	npts x 1			&	the real data series				\\
ntap		&	integer	& input &	1x1				&	number of tapers to avg		\\
		&		&  &					&	0 - default, adaptive method		\\
ntimes	&	integer	& input &	1x1				&	number of iterations		\\
fact	&	real	& input &	1x1				&	degree of smoothing	\\
&		& &					& 	1.0 - default	\\
&		& &					&	range 0.0 to 1.0	\\
nf		&	integer	& input &	1x1				&	number freq. bins		\\
freq		&	real		& output &	nf x 1			&	the real frequency vector			\\
spec		&	real		& output &	nf x 1			&	the spectrum estimate			\\
\hline
\end{tabular}
\end{center}
\label{tab:sine_arg}
\end{table}%

The subroutine also has some optional arguments, including the number of tapers used at each frequency bin (useful for uncertainties), and the approximate 95\% confidence intervals. The call would then be: 
\begin{quote}
\begin{verbatim}
call sine_psd (npts,dt,x,ntap,ntimes,fact,nf,freq,spec,kopt,err)
\end{verbatim}
\end{quote}
Table  \ref{tab:sine_opt} briefly provides information about these two arguments.

\begin{table}[htdp]
\caption{\texttt{sine\_psd} optional arguments}
\begin{center}
\begin{tabular}{|c|c|c|c|l|}
\hline
\textbf{Var} &	\textbf{Type}	&  \textbf{I/O}	&	\textbf{Dimension}	&	\textbf{Description}		\\
\hline	\hline
kopt	&	integer		& output &	nf x 1			&	number of tapers used			\\
err		&	real		& output &	nf x 2			&	95\% confidence intervals			\\
\hline
\end{tabular}
\end{center}
\label{tab:sine_opt}
\end{table}%

\section{Multivariate subroutines}
Various multivariate subroutines are available, for coherence and transfer function estimation and deconvolution. Table \ref{tab:mvar_arg} provides a complete description of the arguments for all multivariate subroutines. To keep the documentation short, I only present examples of simple calls for all multivariate subroutines.

Using Thomson's multitaper method the user can find three subroutines with corresponding calls:
\begin{quote}
\begin{verbatim}
call mt_cohe( npts,dt,xi,xj,tbp,kspec,nf,p,freq,cohe)  
\end{verbatim}
\end{quote}

\begin{quote}
\begin{verbatim}
call mt_transfer(npts,nfft,dt,xi,xj,tbp,kspec,nf,    &
                       freq,cohe,trf)
\end{verbatim}
\end{quote}

\begin{quote}
\begin{verbatim}
call mt_deconv ( npts,nfft,dt,xi,xj,tbp,kspec,nf,tfun)     
\end{verbatim}
\end{quote}           
To use the sine multitaper method for coherence analysis type:
\begin{quote}
\begin{verbatim}
call sine_cohe (npts,dt,xi,xj,ntap,ntimes,fact,nf,p,freq,cohe) 
\end{verbatim}
\end{quote}
For a complete description of all mandatory and optional arguments, consult Table \ref{tab:mvar_arg}.

\vspace{0.5cm}

\begin{quote}
Germ\'an A. Prieto \\
Stanford, California \\
\today
\end{quote}

\begin{sidewaystable}[h!]
\centering{
\caption{Multivariate subroutines arguments}
\begin{tabular}{|c|c|c|c|l|l|}
\hline
\textbf{Var} &	\textbf{Type}	&  \textbf{I/O}	&	\textbf{Dimension}	&	\textbf{Subroutines} & \textbf{Description}		\\
\hline	\hline
npts	&	integer		& input &	1 x 1			& all	& number of points in xi, xj			\\
nfft		&	integer	& input &	1 x 1			& mt\_transfer, mt\_deconv	& allows for zero padding			\\
dt		&	real	& input &	1 x 1			& all	& sampling interval			\\
xi		&	real	& input &	npts x 1			& all	& real data series 1			\\
xj		&	real	& input &	npts x 1			& all	& real data series 2			\\
tbp		&	real	& input &	1 x 1			& all, except sine\_cohe	& time-bandwidth product			\\
kspec		&	integer	& input &	1 x 1			& all, except sine\_cohe	&number of tapers			\\
nf		&	integer	& input &	1 x 1			& all	& number of frequency bins			\\
p		&	real	& input &	nf x 1			& all	& prob. of zero coherence			\\
ntap		&	integer	& input &	1 x 1			& only sine\_cohe	& number of tapers to avg			\\
ntimes		&	integer	& input &	1 x 1			&  only sine\_cohe	& number of iterations			\\
fact		&	real	& input &	1 x 1			&  only sine\_cohe	& degree of smoothing			\\
freq		&	real	& output, optional & 	nf x 1			& all	& real frequency vector			\\
cohe		&	real	& output, optional &	nf x 1			& all	& coherence output			\\
trf		&	complex	& output, optional &	nfft x 1			& all	& transfer function output			\\
tfun		&	real	& output, optional &	nfft x 1			& all	& time domain deconv result			\\
phase		&	real	& output, optional &	nf x 1			& mt\_cohe, sine\_cohe	& phase between two signals			\\
speci		&	real	& output, optional &	nf x 1			& all	& spectrum of signal xi			\\
specj		&	real	& output, optional &	nf x 1			& all	& spectrum of signal xj			\\
conf		&	real	& output, optiona; &	nf x 1			& mt\_cohe, sine\_cohe	& bound for null hypothesis of no coherence			\\
cohe\_ci		&	real	& output, optional &	nf x 2			& mt\_cohe, sine\_cohe	& confidence interval for coherence			\\
phase\_ci		&	real	& output,optional &	nf x 2			& mt\_cohe, sine\_cohe	& confidence interval for phase			\\
cspec		&	complex	& output, optional &	nf x 1			& mt\_transfer	& complex cross-spectrum, positive frequencies only			\\
iadapt		&	integer	& input, optional &	1 x 1			& mt\_transfer, mt\_deconv	& choose either adaptive or constant wieghts			\\
demean		&	integer	& input, optional &	1 x 1			& mt\_transfer, mt\_deconv	& force zero mean in complex transfer function	\\
spec\_ratio	&	real	& output, optional &	nf x 1			& only mt\_deconv	& spectral ratio of two series			\\
fmax		&	real	& input, optional &	1 x 1			& only mt\_deconv	& apply cosine filter with fmax			\\
kopt  	&	integer	& output, optional &	nf x 1			& only sine\_cohe	& number of tapers used			\\
\hline
\end{tabular}
\label{tab:mvar_arg}
}
\end{sidewaystable}%




\small{

\bibliographystyle{natbib}

\bibliography{/homes/german/articles/latex/bibliography}

	}


\end{document} 